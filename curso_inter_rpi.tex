\documentclass{beamer}
\usepackage[spanish]{babel}
\usepackage[dvipsnames]{xcolor}
\usepackage{tikz}
\usepackage{hyperref}
%Declaracion del tema modificado
\usetheme[logo=rpi]{arturo}
%Declaracion de contenedor de imagenes
\graphicspath{{images/}}
%--------------------------------------
\title{Curso básico-intermedio de programación en Raspberry Pi}
\subtitle{Curso intersemestral}
\date{\today}
\author[OACM]{M.I Omar Arturo Castillo Méndez} 

\begin{document}
	
	\begin{frame}[plain]
		\titlepage
	\end{frame}
	
	\begin{frame}{Contenido}
		\tableofcontents
	\end{frame}
	
	\section{Introducción}
	
	\begin{frame}{Introducción}
		\framesubtitle{Examen de diagnostico}
		\begin{itemize}
			\item ¿Qué es una tarjeta Raspberry Pi?
			\item Si conoces algún lenguaje de programación, ¿Cuál o cuáles?
			\item ¿Qué es el lenguaje de programación Python?
			\item ¿Qué bibliotecas de Python conoces?
			\item En la configuración de puertos de una RPi, ¿Qué diferencia hay entre el BCM y BOARD?
			\item Escriba un código para encender un LED en una tarjeta Raspberry Pi usando la configuración de puertos BOARD.
			\item ¿Cuáles son tus expectativas del curso?
		\end{itemize}
	\end{frame}
	\subsection{Generalidades y requisitos}
	\begin{frame}
		\frametitle{Introducción}
		\framesubtitle{¿Qué es una Raspberry Pi?}
		Es una tarjeta de desarrollo, diseñada como una computadora modular con una arquitectura ARM y usando un sistema operativo basado en Linux (Raspbian OS).
		\begin{figure}
			\includegraphics[scale=0.4]{rpiboard}
			\caption{Tarjeta Raspberry Pi 4}
		\end{figure}
		
	\end{frame}	

	\section{Configuración}
	\begin{frame}
		\frametitle{Configuración inicial}
		\framesubtitle{Requisitos}
		
		\begin{tcolorbox}[enhanced, title= Material necesario:]
			\begin{itemize}
				\item Tarjeta Raspberry. Pi(Cualquier versión)
				\item Fuente de alimentación de 5V a 3A.
				\item Tarjeta de almacenamiento micro SD de 64 Gb o superior clase 10 o superior.
				\item Mouse y teclado.
			\end{itemize}
		\end{tcolorbox}
		
			
	\end{frame}

	\begin{frame}
		\frametitle{Configuración inicial}
		\framesubtitle{Instalación de Raspbian}
		
		\begin{tcolorbox}[enhanced, title= Instrucciones:]
				\begin{itemize}
					\item Visitar la siguiente liga: \url{https://www.raspberrypi.com/software/}
					\item Descargar la version correspondiente con su respectivo sistema operativo, y dar click en \textbf{descargar}
					\item Una vez instalado se mostrará la siguiente interfaz de Raspberry Pi Imager.
				\end{itemize}
		\end{tcolorbox}
		
		
	\end{frame}
	
	\begin{frame}
		\frametitle{Configuración inicial}
		\framesubtitle{Instalación de Raspbian}	
		
		\begin{figure}
			\includegraphics[scale=0.35]{imager.png}
			\caption{Interfaz: Raspberry Pi Imager}
		\end{figure}
		
	\end{frame}
	\subsection{Instalación de Raspbian}
	\begin{frame}
		\frametitle{Configuración inicial}
		\framesubtitle{Selección de Tarjeta}
		
		\begin{figure}
			\includegraphics[scale=0.35]{imager2.png}
			\caption{Selección de tarjeta Raspberry Pi}
		\end{figure}
		
	\end{frame}
	
	\begin{frame}
		\frametitle{Configuración inicial}
		\framesubtitle{Selección de sistema operativo}
		
		\begin{figure}
			\includegraphics[scale=0.35]{imager3.png}
			\caption{Selección de sistema operativo}
		\end{figure}
		
	\end{frame}
	
	\begin{frame}
		\frametitle{Configuración inicial}
		\framesubtitle{Selección de unidad de almacenamiento}
		
		\begin{figure}
			\includegraphics[scale=0.35]{imager4.png}
			\caption{Selección de unidad de almacenamiento}
		\end{figure}
		
	\end{frame}
	\subsection{Comandos básicos}
	\begin{frame}
		\frametitle{Configuración inicial}
		\framesubtitle{Comandos básicos de la terminal}
		\begin{tcolorbox}[enhanced, title= Instrucciones:]
			\begin{itemize}
				\item sudo :  Comando para acceder a derechos de superusuario.
				\item cd  : Cambiar de directorio a una ruta especificada.
				\item cd .. : Subir un nivel en la ruta.
				\item ls  : Listar archivos de un fichero/carpeta.
				\item pwd : Mostrar la ruta del carpeta.
				\item mkdir : Crear una carpeta.
				\item nano : Gestor de texto desde la terminal.
				\item cp   : Copiar fichero o archivo hacia una ruta especificada.
				
				
			\end{itemize}
		\end{tcolorbox}
		
	\end{frame}
	
	\begin{frame}
		\frametitle{Configuración inicial}
		\framesubtitle{Comandos básicos de la terminal}
		\begin{tcolorbox}[enhanced, title= Instrucciones:]
			\begin{itemize}
				\item ifconfig : Consultar la información de las interfaces de red.
				\item sudo raspi-config : Entrar a la configuración de la Raspberry Pi
				\item pinout : Muestra la distribución de pines de la tarjeta.
			\end{itemize}
		\end{tcolorbox}
		
		
	\end{frame}
	\subsection{Ajustes de tarjeta RPi}
	\begin{frame}
		\frametitle{Configuración inicial}
		\framesubtitle{Cambiar la configuración de la RPi}
		
		Una vez terminada la instalación, se abre una terminal y se ejecuta el siguiente comando date: 
		\begin{figure}
			\includegraphics[scale=0.25]{daterpi.png}
			\caption{Consulta la fecha del sistema}
		\end{figure}
		Para colocar la fecha actual se ejecutará el siguiente comando en la terminal de la RPi: \textbf{sudo date -s ‘YYYY-MM-DD HH:MM:SS'.}
		
	\end{frame}
	
	\begin{frame}
		\frametitle{Configuración inicial}
		\framesubtitle{Configuración de interfaces}
		Se ejecutará el siguiente comando: sudo raspi-config \newline
		Elegir la opción de \textbf{Interface Options}
		\begin{figure}
			\includegraphics[scale=0.3]{configrpi.png}
			\caption{Opciones de con RPi}
		\end{figure}
	
		
	\end{frame}
	
		\begin{frame}
		\frametitle{Configuración inicial}
		\framesubtitle{Configuración de interfaces}
		Selecciona una por una las interfaces de SSH, SPI, I2C, Serial Port, 1-Wire.
		\begin{figure}
			\includegraphics[scale=0.3]{configrpi2.png}
			\caption{Selección de interfaces de la tarjeta de RPi. }
		\end{figure}
		
	\end{frame}
	\subsection{Definición de IP estática}
	\begin{frame}
		\frametitle{Configuración inicial}
		\framesubtitle{Definir una IP estática}
		Para poder trabajar de manera remota sin la necesidad de usar un monitor externo mediante una terminal SSH(Secure Shell). Ejecutamos el siguiente comando en la terminal "\textbf{sudo nmtui} "
		
		\begin{figure}
			\includegraphics[scale=0.25]{rpissh.png}
			\caption{Administrador configuración de red.}
		\end{figure}
	\end{frame}
	\begin{frame}
	\frametitle{Configuración inicial}
	\framesubtitle{Definir una IP estática}
	Seleccionar el tipo de interfaz a convenir, si esta conectada vía Ethernet o Wi-Fi:
		\begin{figure}
			\includegraphics[scale=0.25]{rpissh2.png}
			\caption{Seleccionar la interfaz a modificar}
		\end{figure}
	\end{frame}
	
	\begin{frame}
		\frametitle{Configuración inicial}
		\framesubtitle{Definir una IP estática}
		En el apartado de IPv4 es donde se va a configurar la IP estática.
		\begin{figure}
			\includegraphics[scale=0.25]{rpissh3.png}
			\caption{Se seleccionó una conexion via Ethernet}
		\end{figure}
	\end{frame}
	\begin{frame}
		\frametitle{Configuración inicial}
		\framesubtitle{Definir una IP estática}
		Para consultar la IP a la cual esta conectada la RPi, vamos a salir del menu anterior y ejecutar en la terminal el siguiente comando: "\textbf{ifconfig}"
		
		\begin{figure}
			\includegraphics[scale=0.25]{rpissh5.png}
			\caption{Informacion de la red mediante el comando ifconfig}
		\end{figure}
		
	\end{frame}
	\begin{frame}
		\frametitle{Configuración inicial}
		\framesubtitle{Definir una IP estática}
		Una vez anotada la información anterior acceder el Network manager de la RPi, y capturar en: \textbf{Address}, \textbf{Gateway} y \textbf{DNS Server}
		\begin{figure}
			\includegraphics[scale=0.25]{rpissh4.png}
			\caption{Network manager de la RPi}
		\end{figure}
	\end{frame}
%----------------------------------------------------
\end{document}